\mytopic{The bayesian way of thinking}

One might argue that our standard way of reasoning consists of two basic elements:
the initial belief we have about a topic and the step of updating this belief when we observe something new.

For example, at first, you would expect a set of dice to be fair when playing a game of Yahtzee. But after observing multiple five-of-a-kind in a row, it is natural to believe that the dice are loaded.
And if you actually expected the other players to be sketchy beforehand, you would come to that conclusion way faster.

This simple idea is formalized in Bayes' Theorem, forming the foundation of many Statistics and Machine Learning methods.
As a result, it is possible to include expert knowledge into a mathematical model.
And even if the used expert knowledge is not totally correct, the model will still be able to overcome that - as long as enough data is observed.


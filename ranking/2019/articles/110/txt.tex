\mytopic{Guess Fast Or Wait!}

Imagine you're in an unfamiliar city and trying to find a new place.
Also, you're using your smartphone for navigation. And because you
have a good sense of direction, you already start walking the way you
think is correct while looking up the way on your phone.

If you guessed the direction correctly, you didn't waste any time
waiting for the result, and can continue on your way. If your guess
was wrong, you have to turn around.

A modern CPU does this as well: When its course of action depends on a
result it would have to wait for, it makes an educated guess and
continues along this path. This is called \textbf{speculative execution}.
If the guess was wrong, the CPU has to discard all the actions it has
done in error. It did not waste any time though, as it would have
spent the time waiting instead. However, if the guess turns out to be
correct, the CPU just gained a little speed boost.


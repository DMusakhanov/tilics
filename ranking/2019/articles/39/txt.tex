\mytopic{Why solve any problem with recursion when iteration works just as well?}

Solving a problem step by step, starting at the bottom and linearly working towards the solution, is called iteration. Recursion will do it the other way around. It will start at the top, search its path down and then hand the obtained results all the way back. Sounds tedious, right? And since every problem solved by recursion could also be solved by iteration way faster, why use recursion at all?

The beauty and usefulness of recursion lies in its intuitive nature. When using iteration, you need to code every step towards the solution yourself. When using recursion, you only need to think about the pattern of the problem. The computer will automatically apply this pattern, again and again, until it reaches the bottom of all the smaller parts of the problem. Then it will combine everything to obtain the solution.

There is no general rule when to use recursion and when not to. For any given problem, simply pick the approach that seems the most natural to you.


\mytopic{Why There Is No Perfect Anti-Virus}

In computer science, simple ``Yes/No'' questions are called \textbf{decision
problems}. A decision problem is called \textbf{decidable} if we can
write a program that can always find an answer for such a problem.

An example for such a decision problem is ``does this program contain a
computer virus''. If this problem was decidable, there could be
another program that could always correctly answer this question with
``Yes'' or ``No''. A mathematician called Henry Gordon Rice showed that
such decision problems about the behavior of another program are
always undecidable, unless the answer is \textbf{always} the same.

As most viruses are hidden inside other software and do not always
expose their real nature, this decision problem can't be solved
according to Rice's theorem.


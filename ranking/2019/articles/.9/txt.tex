\mytopic{Atomic Operations}

Imagine a railway intersection where only one train can pass at a time. If any other train tries to pass the intersection while it's already being used, a crash happens.

The same thing can happen in a computer, where we might think that all things happen ordered and sequentially. But in a world where everything is parallelized this can not be guaranteed and as soon as a file gets changed and read at the same time, things can go wrong.

Similar to a light signal for the railway intersection which would only allow one train to pass at a time, computers have something called \emph{atomic operations}. As the name suggests, these are the smallest possible operations and therefore always done in their entirety with no space for anything in between, thus preventing crashes.


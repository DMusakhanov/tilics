\mytopic{Kleene's Star is not Blah Blah}

Computers are great at repetitive tasks. The most beautiful and
concise way of expressing repetition is Kleene's star, namely the
symbol {\tt *} which stands for \textbf{zero or more repetitions}. In
Computer Science theory, the letter A followed by the Kleene Star,
{\tt A$^{*}$}, either means

\begin{itemize}
        \labelsep 0.5em%
        \labelwidth \leftmargini%
        \addtolength\labelwidth{-\labelsep}
        % \listparindent 5em%   
        \itemindent 1.5em%\leftmargini
        %\advance\leftmargini 1.5em

  \item {\tt ""} (the word of length 0), or
  \item {\tt "A"} (the word consisting of one A), or
  \item {\tt "AA"} (the word consisting of exactly two As), etc
\end{itemize}

\noindent
A related but different concept is a \emph{wildcard} where the star is a
placeholder (instead of a sign of repetition). For example {\tt *.txt}
selects all files that end in ``txt''. Kleene's star is not common in
daily life, but wildcards are: A prominent example is ``blah blah''.


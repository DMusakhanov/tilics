\mytopic{The fixpoint of describing yourself}

What could be easier than to write a program that produces its own
code as output?

\vskip 0.5em
It’s not that easy –- even a well-versed programmer will need to spend
some hours to figure out the trick (which can be different for each
programming language).

\vskip 0.5em
One trick is that the program must contain a description of itself,
inside its own code. Because outputting the self-description also
needs code, the self-description must be somehow compressed.

\vskip 0.5em
But once written, such a program --also called a ``Quine''-- reproduces
itself each time it is run, and this in turn is called a fixpoint.
One could call this the fixpoint of life because any lifeform
(including robots) must master this trick if they wanted to build
themselves.


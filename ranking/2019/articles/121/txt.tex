\mytopic{The fixpoint of describing yourself}

What could be easier than to write a program that produces its own code as output?
It's not that easy -- even a versed programmer will
spend some hours to figure out the trick (which can be different for each
programming language).

One trick is that the program must contain a description of itself,
inside its own code. And because outputting that self-description also
needs code, this self-description must be somehow compressed.

But once written, such a program (also called a ``Quine'') reproduces
itself each time it is run, and this in turn is called a fixpoint.
One could call this the fixpoint of life because any lifeform must
master this trick - including robots if they wanted to build themselves
and become a lifeform.


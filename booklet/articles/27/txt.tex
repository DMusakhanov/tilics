\mytopic{There is no such thing as a free lunch}

In Computer Science we have a wide range of very different problems and accordingly many approaches to tackle them.
We might for example try to find some `optimal' solution for a search problem, but it is unclear which specific algorithm one should use.
So it is only natural to think about algorithms that always provide us the best solution, regardless of the problems subtleties.

However, the `No Free Lunch' theorems state, that no single method works better than any other for all possible problems.
Instead, it is necessary to select the method based on the problem (or data) at hand.
That is, there is always a cost associated with selecting a method and unfortunately there is no such thing as a free lunch.

